%% This is file FCEFyN-paper.tex is the template file for publications
%% in the "Revista de la Facultad de Ciencias Exactas, Físicas y Naturales
%% de la Universidad Nacional de Córdoba, Argentina,
%% https://revistas.unc.edu.ar/index.php/FCEFyN.
%% This file was originally written by Gustavo J. Krause.
%% First revision: 2018-07-17
%% Second revision: 2018-07-26 by Mariano Lizarraga (minor modifications)
%%
%% History:
%%           2018-07-20  First version by Gustavo Krause
%%

% Opciones específicas:
%     esp:   Idioma español (por defecto)
%     eng:   Idioma inglés
%     por:   Idioma portugués
%     blind: Se compila la versión para revisores (se ocultan datos de los autores)


\documentclass[esp]{FCEFyN-class}
%
% Título de la publicación

$if(title)$
\title{$title$}
$endif$

% Título corto para el encabezado (copiar el título principal si no resulta demasiado largo)
\shorttitle{Instrucciones para autores de la Revista de la FCEFyN}
       

\author{$for(author)$$author$$sep$\\$endfor$}
$if(date)$
\date{$date$}
$endif$

% Afiliaciones de los autores
\affil[1]{Nombre de Departamento o Instituto,
Nombre Universidad, Provincia o Estado, País}
\affil[2]{(Otro) Nombre de Departamento o Instituto,
Nombre Universidad, Provincia o Estado, País}
\affil[3]{(Otro) Nombre de Departamento o Instituto,
Nombre Universidad, Provincia o Estado, País}

% Apellido del primer autor del trabajo
\firstauthor{Apellido}

% Datos del autor de contacto
\contactauthor{Nombre A. Apellido}           % Nombre y apellido del autor de contacto
\email{nombre.apellido@email.com}            % Correo electrónico del autor de contacto
\mailingaddress{Dirección postal completa}   % Dirección postal completa
\phonenumber{1234-5678 int. 123}             % Número de teléfono

% Datos de la publicación (serán definidas en la edición)
\thisvolume{XX}
\thisnumber{XX}
\thismonth{MES}
\thisyear{20XX}
\receptiondate{dd/mm/aaaa}
\acceptancedate{dd/mm/aaaa}
\publicationdate{dd/mm/aaaa}


% Coloque aquí sus definiciones particulares
\newcommand{\vect}[1]{\mathbf{#1}}  % vectores


% Iniciar documento
\begin{document}

% Introduzca aquí el resumen en español (o en portugués si utiliza la opción [por])
\resumen{
Este documento brinda una plantilla para la preparación de trabajos originales que desean ser
publicados en la Revista de la Facultad de Ciencias Exactas, Físicas y Naturales de la Universidad
Nacional de Córdoba, Argentina.
Se recomienda que el resumen contenga entre 150 y 200 palabras en un solo párrafo, donde deben
resumirse el contexto, la motivación, la metodología empleada, los aportes más originales, los
resultados y las conclusiones de su trabajo.
No deben incluirse citas bibliográficas y se recomienda no introducir acrónimos ni fórmulas en el
resumen o en el título. No haga referencias a figuras o a tablas. Como recomendación general, escriba
su artículo insertando y eliminado texto a partir de este documento. De esta forma, le será más fácil
respetar los estilos predefinidos.}

% Introduzca aquí las palabras clave en español (o en portugués si utiliza la opción [por])
\palabrasclave{
Primera palabra o frase clave, segunda palabra o frase clave, tercera palabra o frase clave. 
(Coloque entre tres y seis palabras o frases clave separadas por coma, las cuales representan la
temática de su trabajo)}

% Insert here the abstract in English language
\abstract{$abstract$}

% Insert here the keywords of your work in English language
\keywords{
Translate to English the same words and phrases written above.}

% Incluir título, autores, resumen, etc.
\maketitle
\thispagestyle{fancy}
\printcontactdata

$body$

% Incluir las referencias
\insertbibliography{Referencias.bib}

\end{document}

